% This file is iccc.tex.  It contains the formatting instructions for and acts as a template for submissions to ICCC.  It borrows liberally from the AAAI and IJCAI formats and instructions.  It uses the files iccc.sty, iccc.bst and iccc.bib, the first two of which also borrow liberally from the same sources.


\documentclass[letterpaper]{article}
\usepackage{iccc}


\usepackage{times}
\usepackage{helvet}
\usepackage{courier}
\pdfinfo{
/Title (Anything2vec: generalizing word2vec via the use of meaningful contexts)
/Subject (Proceedings of ICCC)
/Author (ICCC)}
% The file iccc.sty is the style file for ICCC proceedings.
%
\title{Anything2vec: generalizing word2vec via the use of meaningful contexts \\
Paper type: Technical Paper (see the CFP for paper types)}
\author{JJ Merelo\\
Departamento de Arquitectura y Tecnología de Computadores\\
Universidad de Granada\\
Granada, Spain\\
jmerelo@ugr.es\\
}
\setcounter{secnumdepth}{0}

\begin{document} 
\maketitle
\begin{abstract}
\begin{quote}
The {\em Proceedings of the International Conference on Computational Creativity} will be compiled from electronic manuscripts submitted by the authors.  This paper provides brief style instructions that will facilitate a high-quality, consistent proceedings.
\end{quote}
\end{abstract}

\section{Introduction}

The {\it ICCC Proceedings} will be produced from electronic
manuscripts submitted by the authors. These must be PDF ({\em Portable
Document Format}) files formatted for 8-1/2$''$ $\times$ 11$''$ paper.

Please see the conference web site 
(accessible from {\small \tt http://computationalcreativity.net/})
for submission instructions, including 
different possible paper types and their lengths.

\subsection{Word Processing Software}

As detailed below, ICCC has prepared and made available a set of
\LaTeX{} macros and a Microsoft Word template for use in formatting
your paper. If you are using some other word processing software, please follow the format instructions given below and ensure that your final paper looks as much like this sample as possible.

\section{Style and Format}

\LaTeX{} and Word templates that implement these instructions
can be retrieved electronically at the conference web site.

\subsection{Layout}

Print manuscripts two columns to a page, in the manner in which these
instructions are printed. The exact dimensions for pages are:
\begin{itemize}
\item left and right margins: 0.75$''$
\item column width: 3.375$''$
\item gap between columns: 0.25$''$
\item top margin---first page: 1.375$''$
\item top margin---other pages: 0.75$''$
\item bottom margin: 1.25$''$
\item column height---first page: 6.625$''$
\item column height---other pages: 9$''$
\end{itemize}

\subsection{Format of Electronic Manuscript}

For the production of the electronic manuscript, you must use Adobe's
{\em Portable Document Format} (PDF). A PDF file can be generated, for
instance, on Unix systems using {\tt pdflatex} 
(or {\tt ps2pdf} from a postscript file) or on Windows systems
using Adobe's Distiller. There is also a website with free software
and conversion services: {\tt http://www.ps2pdf.com/}. For reasons of
uniformity, use of Adobe's {\em Times Roman} font is strongly suggested. In
\LaTeX2e{}, this is accomplished by putting
\begin{quote} 
\mbox{\tt $\backslash$usepackage\{times\}}
\end{quote}
in the preamble.
  
Additionally, you must specify the American {\bf
letter} format (corresponding to 8-1/2$''$ $\times$ 11$''$) when
formatting the paper.

\subsection{Title and Author Information}

Center the title on the entire width of the page in a 15-point bold
font. Below it, center the author name(s) in a 12-point bold font, and
then center the address(es) in a 10-point regular font. Credit to a
sponsoring agency can appear on the first page as a footnote.

\subsubsection{Blind Review}

ICCC may use a double blind review process where the author information
must (or may) be hidden in the submissions. 
Please see the call for papers for instructions on possible blind reviews. 

\subsection{Abstract}

Place the abstract at the beginning of the first column 3$''$ from the
top of the page, unless that does not leave enough room for the title
and author information. Use a slightly smaller width than in the body
of the paper. Head the abstract with ``Abstract'' centered above the
body of the abstract in a 10-point bold font. The body of the abstract
should be 9-point in the same font as the body of the paper.

The abstract should be a concise, one-paragraph summary describing the
general thesis and conclusion of your paper. A reader should be able
to learn the purpose of the paper and the reason for its importance
from the abstract. As a guide, abstracts should normally be up to 250 words long.

\subsection{Text}

The main body of the text immediately follows the abstract. Use
10-point type in {\em Times Roman} font.

Indent when starting a new paragraph, except after major headings.

\subsection{Headings and Sections}

When necessary, headings should be used to separate major sections of
your paper. (These instructions use many headings to demonstrate their
appearance; your paper should have fewer headings.)

\subsubsection{Section Headings}

Print section headings centered, in 12-point bold type in the style shown in
these instructions. Leave a blank space of approximately 10 points
above and 4 points below section headings.  Do not number sections.

\subsubsection{Subsection Headings}

Print subsection headings left justified, in 12-point bold type. Leave a blank space
of approximately 8 points above and 3 points below subsection
headings. Do not number subsections.

\subsubsection{Subsubsection Headings}

Print subsubsection headings inline in 10-point bold type. Leave a blank
space of approximately 6 points above subsubsection headings. Do not
number subsubsections.

\subsubsection{Special Sections}

You may include an unnumbered acknowledgments section, including
acknowledgments of help from colleagues, financial support, and
permission to publish.

Any appendices directly follow the text and look like sections.  In general, appendices should 
be avoided in ICCC manuscripts.

The references section is headed ``References,'' printed in the same
style as a section heading. A sample list of
references is given at the end of these instructions.  Note the various examples for books, proceedings, multiple authors, etc. Use a consistent
format for references, such as that provided by BiB\TeX{}. The reference
list should not include unpublished work.

\subsection{Citations}

Citations within the text should include the author's last name and
the year of publication, for example~\cite{boden92}.  Append
lowercase letters to the year in cases of ambiguity.  Treat multiple
authors as in the following examples:~\cite{lyu04}
(for more than two authors) and
\cite{veale07} (for two authors).  If the author
portion of a citation is obvious, omit it, e.g.,
Woods~\shortcite{Woods81}.  Collapse multiple citations as
follows:~\cite{UCI,Ruch07,OZ}.
\emph{See below for \LaTeX{} commands for producing citations.}

\subsubsection{Using \LaTeX{} and BiBTeX to Create Your References}
At the end of your paper, you can include your reference list by using the following commands (which will insert a heading \textbf{References} automatically):

\begin{footnotesize}
\begin{verbatim}
\bibliographystyle{iccc}
\bibliography{bibfile1,bibfile2,...}
\end{verbatim}
\end{footnotesize}

The list of files in the bibliography command should be the names of your BiBTeX source files (that is, the .bib files referenced in your paper).

The iccc.sty file includes a set of definitions for use in formatting references with BiBTeX. These definitions make the bibliography style fairly close to the one specified previously. To use these definitions, you also need the BiBTeX style file iccc.bst available in the author kit on the ICCC web site.

The following commands are available for your use in citing references:
\begin{description}
\item \verb+\+cite: Cites the given reference(s) with a full citation. This appears as ``(Author Year)'' for one reference, or ``(Author Year; Author Year)'' for multiple references.
\item \verb+\+shortcite: Cites the given reference(s) with just the year. This appears as ``(Year)'' for one reference, or ``(Year; Year)'' for multiple references.
\item \verb+\+citeauthor: Cites the given reference(s) with just the author name(s) and no parentheses.
\item \verb+\+citeyear: Cites the given reference(s) with just the date(s) and no parentheses.
\end{description}

{\bf Warning:} The iccc.sty file is incompatible with the hyperref package. If you use it, your references will be garbled. {\it Do not use hyperref!}

\subsection{Footnotes}

Avoid footnotes as much as possible; they interrupt the flow of
the text. 
Place footnotes at the bottom of the page in 9-point font.  Refer to
them with superscript numbers.\footnote{This is how your footnotes
should appear.} Separate them from the text by a short
line.\footnote{Note the line separating these footnotes from the
text.} 

\section{Illustrations}

Place all illustrations (figures, drawings, tables, and photographs)
throughout the paper at the places where they are first discussed (at the bottom
or top of the page), rather than at the end of the paper. If placed at the bottom or top of
a page, illustrations may run across both columns.

Illustrations must be rendered electronically or scanned and placed
directly in your document. In most cases, it is best to render all illustrations
in black and white; however, since the proceedings are produced and distributed electronically, if color
is important for communicating your message, it may be included. Line weights should
be 1/2-point or thicker.

Number illustrations sequentially. Use references of the following
form: Figure 1, Table 2, etc. Place illustration numbers and captions
under illustrations. Leave a margin of 1/4-inch around the area
covered by the illustration and caption.  Use 9-point type for
captions, labels, and other text in illustrations.

\section{Acknowledgments}

The preparation of these instructions and the \LaTeX{} and Word files was 
facilitated by borrowing from similar documents used for AAAI and IJCAI proceedings.


%\appendix{\LaTeX{} and Word Style Files}\label{stylefiles}

%The \LaTeX{} and Word style files are available on the ICCC-13
%website, {\tt http://computationalcreativity.net/iccc2013/}.
%These style files implement the formatting instructions in this
%document.

%The \LaTeX{} files are {\tt iccc.sty} and {\tt iccc.tex}, and
%the Bib\TeX{} files are {\tt iccc.bst} and {\tt iccc.bib}. The
%\LaTeX{} style file is for version 2e of \LaTeX{}, and the Bib\TeX{}
%style file is for version 0.99c of Bib\TeX{} ({\em not} version
%0.98i).

%The Microsoft Word style file consists of a single template file, {\tt
%iccc.dot}. 

%These Microsoft Word and \LaTeX{} files contain the source of the
%present document and may serve as a formatting sample.  


\bibliographystyle{iccc}
\bibliography{iccc}


\end{document}
